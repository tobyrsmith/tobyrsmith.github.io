\documentclass[12pt]{article}

\pagestyle{empty}

\setlength{\oddsidemargin}{-1cm}
\setlength{\evensidemargin}{-1cm}
\setlength{\textwidth}{18cm}
\setlength{\textheight}{24cm}
\setlength{\footskip}{0in}
\setlength{\voffset}{-2cm}

\newenvironment{Quote}
               {\list{}{\setlength{\rightmargin}{5pt}%
                        \setlength{\leftmargin}{5pt}
                        \setlength{\baselineskip}{12pt}}%
                \item\relax}
               {\endlist}


\begin{document}

\parindent 0pt


{\large \bf Books} 

\vskip 2mm

There are literally tons of books about the Apollo program and the
Moon.  If you want to waste a rainy afternoon wandering through the
Apollo literature head over the forth floor of the Engineering
library, they have several shelves of Apollo and other mission
literature.  Below I have listed my favorite books about Apollo.  Most
of them have extensive bibliographies if you want to explore further.

%\vfill
\vskip 5mm

{\bf Apollo: The Race to the Moon} by Murray, C. and Cox, C. B.  New
York: Simon and Schuster, 1989. 506pp.
\begin{Quote}
  My favorite general Apollo book, and probably the best introduction
  to the Apollo missions.  Told mainly from the view point of the
  people in mission control and the engineers.  This book actually
  makes the engineers' lives sound exciting.  Great accounts of Apollo
  8 and 13.  Newly reissued.
\end{Quote}

\vfill

{\bf To a Rocky Moon: A Geologist's History of Lunar Exploration} by
Wilhelms, D. E.  Tucson: University of Arizona Press, 1993. 477pp.
\begin{Quote}
  Written by a scientist heavily involved in the Apollo missions and
  lunar science in general.  Everything you need to know about what
  Apollo taught us about the Moon.  Lots of early history about lunar
  science, training the Apollo astronauts, choosing landing sites, and
  the role of scientists during the Apollo missions.  The book can get
  a little technical at times (mostly in the beginning) but generally
  is it worth the effort.  Out of print, but easy to find used online.
\end{Quote}

\vfill

{\bf Carrying the Fire: An Astronaut's Journeys} by Collins, M.  New
York: Farrar, Straus, and Giroux, 1974. 478 pp.
\begin{Quote}
  Written by the Command Module Pilot of Apollo 11.  Easily the best
  book written by an astronaut.  The best place to learn what is was
  like being an astronaut and going to the Moon.  The book is a bit
  dated now but no other first-hand account of space flight has come
  close to this one.  Newly reissued.
\end{Quote}

%{\bf Apollo Expeditions to the Moon} edited by Cortright, E. M.  NASA,
%1975. 313 pp.
%\begin{Quote}
%  A collection of chapters written by all the heavy hitters in the
%  Apollo program.  The text is OK but the real selling point is that
%  it has lots of great pictures of all the aspects of the Apollo
%  missions.
%\end{Quote}

\vfill

{\bf The Lunar Sourcebook: A User's Guide to the Moon} edited by
Heiken, Vaniman, and French.  Cambridge: Cambridge University Press,
1991. 736 pp.
\begin{Quote}
  The Moon's reference book.  If you need to know something about the
  Moon, its history, geology, chemistry, or what materials are
  available to build a colony it is in this book.  It can get {\em
  really} technical but if you need to know this is the book to turn
  to.  Used copies are really expensive, but there is a \$30 cd-rom
  version available online.
\end{Quote}

\vskip 5mm

{\bf National Geographic} - The ubiquitous yellow bordered magazine
found at every used book sale.  National Geographic published nice
summaries of most of the Apollo missions, all with great pictures.

\begin{Quote}
  
  {\sc February 1969} - Pre-Apollo look at the Moon - lots of great
  Lunar Orbiter images.  Also includes a nice fold-out Moon Map.
  
  {\sc May 1969} - Apollo 8
  
  {\sc December 1969} - Apollo 11 - A classic, even has a cool record
  you can play.  That is if you can find a turntable.
  
  {\sc July 1971} - Apollo 14
  
  {\sc February 1972} - Apollo 15
  
  {\sc December 1972} - Apollo 16
  
  {\sc September 1973} - Apollo 17
\end{Quote}

\normalsize

\clearpage

\end{document}
