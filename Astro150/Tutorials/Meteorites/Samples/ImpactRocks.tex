%\documentclass[12pt]{book}
\usepackage{graphicx}
\usepackage{booktabs}
\usepackage{lab}
\usepackage{fancyhdr}
\usepackage{shadow}
\usepackage{array}
%\usepackage[Sonny]{fncychap}
\usepackage{wrapfig}
\usepackage{multicol}

\newcounter{Lab}
\setcounter{Lab}{-1}

\newcommand{\Labtitle}[1]{
\stepcounter{Lab}
\noindent
%\rule{\linewidth}{1pt}\\[5pt]
\textsc{\large Lab \#\theLab \hfill #1}\\
\rule{\linewidth}{1pt}
\vspace*{-5mm}
}


\pagestyle{fancy}
\fancyhf{}%
\lfoot{\scshape Astronomy 150}%
\cfoot{\theLab\ -- \thepage}%
\rfoot{\scshape The Planets}%
\renewcommand{\headrulewidth}{0pt}

\newcommand{\s}{\rule[-4mm]{0cm}{9mm}}



\makeatletter
\renewcommand{\section}{\@startsection{section}{1}{0pt}%
                       {-0.5\baselineskip}{0.5\baselineskip}%
                       {\normalfont\large\bfseries}%
}
\makeatother

\begin{document}




\setlength{\parindent}{0em}
\setlength{\baselineskip}{15pt}
\setlength{\parskip}{0.6ex}
\setcounter{page}{1}
%\setcounter{Lab}{7}

\Labtitle{Impactors and the Impacted}

\vfill

\section*{Introduction}

Meteorites are fragments of other worlds that have survived the entry
into the Earth's atmosphere.  Most meteorites originate in the
asteroid belt from bodies that formed very early in the history of the
solar system.  Almost all of the information we have learned about the
solar system, such as its age, history, and chemical composition, is
due to the detailed study of meteorites.  From the point of view of
origin, there are three basic types of meteorites: {\bf stony}, {\bf
  stony-iron}, and {\bf iron}.  Meteoriticists recognize many more
types of meteorites and have reconstructed a marvelously detailed
history of the solar system from their subtle differences.

When a large meteorite strikes the Earth, the kinetic energy of the
meteorite is converted into thermal, mechanical, and acoustic energy
that creates a shockwave that passes through the ground and distorts,
fractures, and ejects pieces of the target.  This modified target
material is often all that remains of a crater after millions of years
of geological activity.  Therefore, the recognition of this material
plays an important role in understanding impact events.  The most
common types of impact--modified material we will see are: {\bf impact
  breccia}, {\bf shatter cones}, and {\bf tektites}.


\vfill

\subsection*{Iron Meteorites}

\begin{wrapfigure}{r}{9cm}
\includegraphics[width=9cm]{./images/Iron_lab.ps}
\end{wrapfigure}

Iron meteorites are the most easily recognizable meteorites.  Since even
a casual examination shows that they are not ordinary rocks, they tend
to be very common in collections although they are rare in space.
They are very dense and, except for a thin crust (made by the melting
of the skin during the passage through the atmosphere), they look and
feel like metal.  Chemically, they are composed mostly of iron with a
few percent nickel and a little cobalt.  When sawed in half, polished,
and etched, they display a geometrical pattern called a {\bf
  Widmanst\"atten} pattern (see figure).  This pattern is actually
crystals of iron and nickel that form as the result of the meteorite
having cooled very slowly (about 1$^{\circ}$ per million years) under
very high pressure.  The existence of the {\bf Widmanst\"atten}
pattern is our best evidence that iron meteorites were once the cores
of larger, differentiated bodies.  Buried deep in a body, the mass of
the overlying rocks provide the high pressure and the insulation for
slow cooling.  [The image shows a polished and etched cross--section
of an iron meteorite from the Henbury impact craters in Australia.]


\clearpage

\subsection*{Stony Meteorites}

\begin{wrapfigure}{r}{8cm}
\includegraphics[width=8cm]{./images/Stony_lab.ps}
\end{wrapfigure}

Stony meteorites are the most common meteorites that fall to Earth.
Since they tend to have a similar appearance and density as Earth
rocks, stony meteorites are difficult to recognize in the field.
Unless someone sees them fall, they usually go uncollected.
Therefore, although stony meteorites are the most common type out in
space, they are more rare than iron meteorites in collections on
Earth.  Stony meteorites show a wide variety of appearances: some
light, some dark, some coarse grained, some fine--grained, but almost
all stony meteorites contain some metallic iron.  Chemically they are
also diverse, though they all have a telltale composition that tells
us they are not from the Earth.  Most stony meteorites are from the
outer parts of an asteroid that suffered destruction by collision.
Some are pieces of lava flows from the surface (Achondrites), some are
pieces of impact breccia (Achondrites), and some are pieces of
material that apparently never existed in a much larger body (Ordinary
and Carbonaceous Chondrites).  Meteorites that come from such a small,
undifferentiated body are called primitive meteorites.  [This image
shows a slice of a type of stony meteorite called an {\it ordinary
chondrite}.  This sample fell to the Earth in Homestead, Iowa on
February 12, 1876.]


\vfill

\subsection*{Stony-Iron Meteorites}

\begin{wrapfigure}{r}{8cm}
\includegraphics[width=8cm]{./images/StonyIron_lab.ps}
\end{wrapfigure}

Stony-Iron meteorites are the rarest class of meteorites, comprising
only about 1\% of meteorites that fall to Earth.  There are two broad
classes of stony-iron meteorites: {\it Pallasites} ( composed
primarily of iron with crystals of a rock mineral called olivine
embedded in it) and {\it mesosiderites} (that look like stony
meteorites with lots of metallic iron veins running through them).
{\it Pallasites} are thought to be material from the boundary zone
between the iron cores and the stony outer mantles of the
now-destroyed asteroids, while {\it mesosiderites} are theorized to be
formed when an impact on an asteroid mixes material from the rocky
mantle with iron from the core.  [The image shows a polished slice of
a {\it pallasite} stony-iron meteorite.  The dark roundish inclusions
are the rocky mineral {\it olivine}, and the lighter surrounding
material is metallic iron.  This sample is from the Brenham meteorite
crater in Kansas.]

\vfill

\clearpage

\subsection*{Carbonaceous Chondrite Meteorites}


\begin{wrapfigure}{r}{8cm}
\includegraphics[width=8cm]{./images/CC_lab.ps}
\end{wrapfigure}
%
An especially important type of meteorite is the carbonaceous
chondrite, a specific type of stony meteorite that originates from
primitive asteroids.  They are black to dark gray in color, rich in
the element carbon (thus their black color), and contain small
spherical droplet-like inclusions called chondrules.  They are among
the most primitive objects in the solar system, having survived almost
unchanged for 4.6 billion years.  Carbonaceous chondrites were the
first place amino acids were found outside of the Earth, and it has
been recently learned that some of the materials in these meteorites
were formed outside of our solar system before our solar system was
even formed, so they are not only an important probe into our early
solar system history, but they may supply us with samples of materials
from beyond our solar system.  Although carbonaceous chondrites are
fairly abundant among meteorites that fall to the Earth, they look
enough like Earth rocks that they are rare in collections.  They also
weather very easily and do not survive long on the surface of the
Earth.  [This image shows a slice of a type of carbonaceous chondrite
meteorite that fell to the Earth in Allende, Mexico, on February 9,
1969.]

\vfill

\subsection*{Impact Breccia}

\begin{wrapfigure}{r}{8cm}
\includegraphics[width=8cm]{./images/Breccia_lab.ps}
\end{wrapfigure}

Impact breccias form when a crater--forming meteorite shatters,
pulverizes, and melts the target material.  They are composed of rock
and mineral fragments embedded in a {\bf matrix} of fine--grained
material.  The fragments are usually sharp and angular, and vary
greatly in size and shape.  The composition of the fragments depends
on the target material.  Impact breccias often have the appearance of
poorly mixed concrete (see figure).  On airless, impact--covered worlds
like the Moon, impact breccias are a very common type of rock.  The
most common type of sample returned by the Apollo lunar mission was
impact breccia.  Unfortunately, rocks that look a lot like impact
breccias can be formed by volcanic and tectonic processes, so finding
a breccia is not always a clear indication of an impact event.  [This
images shows a piece of impact breccia from the Ries crater in
Germany.]

\vfill

\clearpage

\subsection*{Shatter Cones}

\begin{wrapfigure}{r}{6cm}
\includegraphics[width=6cm]{./images/ShatterCone_lab.ps}
\end{wrapfigure}

Shatter cones form when the shockwave from a meteorite impact event
passes through the target rocks and modifies them.  The resulting
rocks have distinctive, curved, striated fractures that typically form
partial to complete cones [see figure].  Shatter cones can form in all
types of target rocks.  The better--looking shatter cones form in very
fine--grained rocks like sandstones.  They can range in size from
centimeters to many tens of meters.  Shatter cones are now accepted as
a unique identifier of a meteorite impact event.  This means that if
you find a shatter cone, you have found a place where a meteorite has
hit.  Since the Earth is such an dynamic world, it will erase impact
craters over a relatively short time period.  Often, shatter cones are
all that is left to identify an impact crater.  An interesting feature
of shatter cones is that the tips point toward the origin of the
shockwave.  This means that you can use shatter cones to reconstruct
the size and shape of ancient impact craters that have subsequently
been modified by other processes.  [This image shows a shatter cone
from the Steinheim Basin in Germany.]

\vfill

\subsection*{Tektites}

\begin{wrapfigure}{r}{6cm}
\includegraphics[width=6cm]{./images/Tektite_lab.ps}
\end{wrapfigure}

Tektites have been controversial objects since their discovery, with
both their origin and source being a subject of hot debate for more
than a century.  Tektites are small, glassy objects with shapes like
spheres, ellipsoids, dumbbells, and other forms characteristics of
isolated molten blobs.  They are typically black, but can be brown,
gray, or even green.  Tektites look a lot like volcanic glass ({\it
  e.g.,} obsidian) but are chemically distinct.  The most telling
chemical difference is that unlike volcanic glasses, tektites contain
virtually no water.  Current scientific consensus is that tektites are
terrestrial material that has been melted and ejected from an impact
event.  Their shape is derived from cooling, aerodynamically, during
flight from the impact.  Tektites are fairly common all over the
Earth.  (You can almost always find them for sale in ``New Age''
crystal shops.)  However, linking them with particular impact events
has proven problematic.  When exactly the tektites are formed during an
impact event, and why they are found at only a few craters are two of
the more obvious problems that have yet to be satisfactorily solved.
[This image shows a collection of tektites from Thailand.]

\vfill

\clearpage

\Top

%\vskip 1cm

\vfill

The tables in the lab have a number of different types of samples.
Examine them carefully, with the idea that afterwards you will be
identifying samples for which the types are not going to be given.
[You will probably also see one of these on the final exam.]

\vfill

\footnotesize

\begin{flushleft}
\begin{tabular}%
  {|>{\PBS\centering\hspace{0pt}}m{20mm}%
    |>{\PBS\centering\hspace{0pt}}m{25mm}%
    |>{\PBS\centering\hspace{0pt}}m{45mm}%
    |>{\PBS\centering\hspace{0pt}}m{60mm}|}%

\hline

\rule[0mm]{0cm}{8mm} {\bf Sample Type}  &
\multicolumn{3}{c|}{\bf Specific Characteristics} \\
\cline{2-4}

\rule[0mm]{0cm}{5mm} & {\bf  Density} & 
{\bf  Color \& Textures} & {\bf  Other}\\ \hline

Iron Meteorite & \rule[-6mm]{0cm}{20mm} & &  \\ \hline
Stony Meteorite & \rule[-6mm]{0cm}{20mm} & &  \\ \hline
Stony-Iron Meteorite & \rule[-6mm]{0cm}{20mm} & &  \\ \hline
Carbonaceous Chondrite & \rule[-6mm]{0cm}{20mm} & &  \\ \hline
Impact Breccia & \rule[-6mm]{0cm}{20mm} & &  \\ \hline
Shatter Cone & \rule[-6mm]{0cm}{20mm} & &  \\ \hline
Tektite & \rule[-6mm]{0cm}{20mm} & &  \\ \hline

\end{tabular}
\end{flushleft}

\vfill

\clearpage

\normalsize

{\bf \large Unknown Samples}

\vspace*{2em}

On the table are a number of unknown samples.  Write down what type of
meteorite or impact rock you think it is and why.  One or more of the
rocks might not be meteorites or impact rocks at all.  If you think
one of the samples is not a meteorite or impact rock, just write down
ROCK as its type.

\vskip 1cm

\begin{tabular}{ccc}
{\bf Unknown} & \ \ \ \ \ \ \ {\bf Sample Type} & \ \ \ \ \ \ \ \ {\bf Reason for this Identification} \\
  & \Space{5mm}& \\
  1. & \Space{1.5cm} & \\ 
  2. & \Space{1.5cm} & \\ 
  3. & \Space{1.5cm} & \\ 
  4. & \Space{1.5cm} & \\ 
  5. & \Space{1.5cm} & \\ 
  6. & \Space{1.5cm} & \\ 
  7. & \Space{1.5cm} & \\ 
  8. & \Space{1.5cm} & \\ 
\end{tabular}

\clearpage
\normalsize

%\end{document}
